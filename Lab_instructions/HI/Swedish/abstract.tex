\chapter*{Sammanfattning}
\addcontentsline{toc}{chapter}{Sammanfattning}
SALSA-Onsala (``Sicken Attans Liten Söt Antenn'') är ett 2.3m diameter
radioteleskop utvecklat av Onsala Rymdobservatorium för att introducera elever,
studenter och lärare till radioastronomi. Den känsliga mottagaren gör det
möjligt att detektera bland annat radiovågor från atomär vätgas lång bort i vår
galax Vintergatan. Med hjälp av mätningarna så kan vi lära oss om hur gasen i
galaxen roterar och skapa en karta som visar galaxens spiralstruktur.

I detta dokument så beskriver vi först det galaktiska koordinatsystemet, 
hur radiovågorna från väte uppkommer och hur vi kan relatera våra mätningar
till hastighet genom Dopplerskift. Sedan beskriver vi hur SALSA kan användas
för att förstå hur snabbt vätgasen roterar på olika avstånd från galaxens centrum,
d.v.s. vi skapar en \emph{rotationskurva}. Slutligen så visar vi hur du,
med hjälp av rotationskurvan, kan skapa en karta över Vintergatan.

Observera att detta dokument fokuserar på den vetenskapliga förståelsen
av mätningarna. Instruktioner för att styra teleskopet och göra mätningar
finns i dokumentet \emph{SALSA bruksanvisning} som du hittar på SALSA-hemsidan.

\vspace{9cm}




{\bf Omslagsbild:} En konstnärs bild av Vintergatans spiralstruktur. Avstånd
är i ljusår (ly) och riktningar i galaktiska koordinater. Källa:
NASA/JPL-Caltech/ESO/R. Hurt.
