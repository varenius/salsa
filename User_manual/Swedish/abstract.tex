\chapter*{Sammanfattning}
\addcontentsline{toc}{chapter}{Sammanfattning}
SALSA-Onsala (``Sicken Attans Liten Söt Antenn'') är ett 2.3m diameter
radioteleskop utvecklat av Onsala Rymdobservatorium för att introducera elever,
studenter och lärare till radioastronomi. Den känsliga mottagaren gör det
möjligt att detektera bland annat radiovågor från atomär vätgas lång bort i vår
galax Vintergatan. Med hjälp av mätningarna så kan vi lära oss om hur gasen i
galaxen roterar och skapa en karta som visar galaxens spiralstruktur.

I detta dokument så beskriver vi i detalj hur du styr teleskopet och hur du
kan analysera den data som du får ut från SALSA. Här finns också en kort
sammanfattning över teleskopets tekniska möjligheter och begränsningar.

Vänligen observera att detta dokument inte fokuserar på någon vetenskaplig
tolkning av mätresultat. Mer information om hur du kan tolka dina mätningar
finns i de dokument på SALSA-hemsidan som beskriver olika projekt, till exempel
projektet \emph{Kartläggning av Vintergatan}.

\vspace{9cm}




{\bf Coverimage:} Foto av SALSA-teleskopen i Onsala.
