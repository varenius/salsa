\chapter{Observing with SALSA}
It is important to be well prepared before you start observing with SALSA.
Time is not only limited, because of the Earth's rotation some objects on the
sky can only be seen during a specific part of the day.  Please try to get a
clear understanding of what you want to do before using the telescope.  In this
chapter we describe what can be observed, how to control the telescope, and how
to extract your measurements. We strongly recommend you to read this document
before starting your first observations. 

\section{What can we observe with SALSA?}
Although the SALSA system was primarily designed for observing galactic
hydrogen there are a few other ways to use the telescope. In the following
subsections we describe the ideas we have tested so far. 

\subsection{Galactic hydrogen}
Most users of SALSA aim to detect emission from hydrogen gas in our galaxy the
Milky Way.  The aim here is to detect emission from hydrogen gas emitting at
frequencies close to 1420.4\,MHz, the rest frequency of hydrogen.  This is what
the SALSA website and control program was designed for and most of the
available documentation is related to this particular task. For an extended
description of how to perform and understand this task, see the document 
\emph{Mapping the Milky Way}.

\subsection{Andromeda?}

\subsection{Beam size of SALSA?}

\section{Connecting to SALSA} 
\label{sect-connect}
The SALSA telescopes are controlled from a computer in Onsala. If you are at
the observatory, then you can login to the computer directly. However, most
observations are done remotely via internet. To control SALSA you thus need to
login remotely to a computer in Onsala. This remote control has been tested
on Windows, Mac OS X and Linux so you should be able to connect using almost
any computer. There are two common ways to connect to the computer in Onsala:

\begin{itemize}
	\item{\emph{The free software NoMachine.} This is the most common way to control a
SALSA telescope. Before your first observations, you will need to download
the free software NoMachine from  www.nomachine.com and install it on
your computer.  The first time you connect using the NoMachine software you may
follow the step-by-step guide in Appendix \ref{app:nomachine}.} Once you are
connected you may start the control program by clicking on the SALSA-shortcut
on the virtual desktop.
\item{\emph{The terminal.} If you are used to working in the terminal you may login
	through SSH with graphics support using the command "ssh -X
username@computer".  You may then start the control program from the terminal
with the command {\tt  SALSA}.}
\end{itemize}

\subsection{Ending your session}
When you are done observing, please close the control program using the
\emph{x} in the upper right corner. Then close your connection to the SALSA
computer. If you are connected using NoMachine, you close your connection by
closing the NoMachine window (e.g. by clicking the \emph{x} in the upper corner) 
and then chose {\tt Terminate} when asked how to quit the
session. If you are connected via SSH in a terminal, you close your connection by 
typing {\tt exit}.

\subsection{Troubleshooting}
Do you have trouble connecting to the telescope? Before contacting support,
please check the three most common issues:
\begin{itemize} 
\item The password for the control computer may be different from the password
you chose for the webpage (to make bookings). To log in to a control computer
you need to use your \emph{telescope password} which you can find under
\emph{my account} on the SALSA website.
\item Make sure you connect to the right computer. The host adress is different
for different telecopes, but it has always the same format. For example, if
you have booked the telescope \emph{vale} then the computer is
\emph{vale.oso.chalmers.se}. If you have booked the telescope \emph{brage}
then the computer is \emph{brage.oso.chalmers.se}. 
\item Make sure you have made a reservation at the correct time. The booking
	system shows all times in your selected timezone. You may check the time
	right now in your selected timezone by looking at the clock on the SALSA
	website. 
\end{itemize}

\section{The telescope control program} 
\label{sect-control}
When you are logged in on a telescope computer you can start the control
program, by either clicking the icon \emph{SALSA} on your desktop, or by
running \emph{SALSA} in a terminal.  You should \footnote{Sometimes the window
	takes more than 10 seconds to appear, be patient. If you do not see any
window, try again and wait up to 30 seconds. If you still see nothing, please
contact support as described on the SALSA website.} now see the main control
program looking very similar to Fig. \ref{fig:controlstart}. 
\begin{figure}[ht]
\begin{center}
\includegraphics[width=\textwidth]{../figures/Controller_start.png}
\end{center}
\caption{Startup display of the SALSA control program.}
\label{fig:controlstart}
\end{figure}

The control program is used to move the telescope and to record measurements.
We will start by describing how to move the telescope, and then we describe
how to record data.

\subsection{Movement control: To point the telescope}
The startup display of the SALSA control program contains a box labelled
\emph{Telescope movement control}. This box contains four rows of white
fields. We now describe the purpose of these fields in detail. 

The first two rows are for user input, i.e. you may enter values here.  The
first row is labelled \emph{Desired}. This row must be specified by you, this
is where you specify where you want the telescope to point. Different
coordinate systems are valid, but Galactic coordinates are most common since
they are used when observing galactic hydrogen. A few special objects can also
be selected directly, for example the Sun.

The second row is labelled
\emph{Desired horizontal offset}. This row is only used in special cases, for
example when doing beam measurments, and should be left at 0 for most
observations, e.g. for galactic hydrogen. 

The rows three and four are for display only, i.e. you do not enter any values
here.  The third row is labelled \emph{Calc. target horizontal}. This row
displays the target local (altitude-azimuth) coordinates as calculated by the
control program given the desired coordinates you have entered (including a
possible offset from the second row). Note that the coordinates are changing
as the current pointing is re-calculated every second. These calculations
are done automatically by the program. 

The fourth and last coordinate row shows the current local coordinates, i.e.
where the antenna is pointing at this very moment. Once you tell the antenna
to move, see below, it will start moving until the current coordinates (row four)
is the same as (or very close to) the position calculated in row three.

\subsubsection{Tracking}
Tracking means to track or \emph{follow} a specific object or coordinate on the
sky. This means that the telescope needs to move to correct for the movement of
the Earth (the rotation is about 0.25$^\circ$ per minute). Once you have
specified the target coordinates correctly, click on the button {\tt Track}.
The telescope will now start moving, see Fig. \ref{fig:controlmove}, and will
keep calculating and moving to follow your target on the sky until you tell it
to stop by pressing the button {\tt Stop}.  Don't forget to look at the webcam
at the SALSA website to check that the telescope is indeed moving.  Once you
reach the target you will probably not notice the minor tracking movement by
eye, but if you look carefully you will see the \emph{Current} coordinates will
changing slightly over time to follow the change in the calculated position.
\begin{figure}[ht]
\begin{center}
\includegraphics[width=\textwidth]{../figures/Controller_move.png}
\end{center}
\caption{The telescope control program display when the telescope is moving.}
\label{fig:controlmove}
\end{figure}

\subsubsection{Close enough to measure}
Note that it may take up to a few minutes to reach your desired position if you
started pointing far away on the sky. Measuring while moving will produce
nonsense data.  Please wait until the telescope is within 1 degree of your
desired position before you measure anything.  The control program will assist
you by changing the background color of the \emph{Current} coordinates from
yellow (meaning still not close enough to measure, see e.g. Fig.
\ref{fig:controlmove}) to white (meaning close enough to measure).  Further
information on positional accuracy can be found in Sect. \ref{sect:tech}.

\subsubsection{Keep above 15$^\circ$ altitude}
The telescope will refuse to move if you give it unreachable position, and you
will be informed what the allowed limits are (i.e. you cannot break it).
However, although the telescope can move down to the horizon it is wise to only
measure at high enough altidudes to avoid disturbing radio emission from the
Earth itself.  As a rule of thumb, make sure that the target altitude is larger
than 15 degrees.

\subsection{Receiver control: To measure a spectrum}
When the telescope has reached a desired target coordinate you are ready to
measure a spectrum. It is important to keep tracking during the measurement, so
do not stop the tracking until your measurement is finished.  Before starting a
measurment you need to decide for how long you want to measure. A longer time
means a clearer signal. The measurement time is called \emph{integration time}
and you find it in the box marked \emph{Receiver control} in the middle of the
control program window.  One usually obtains a good spectrum of galactic
hydrogen after 20 seconds.  After entering the integration time, click on {\tt
Measure}. You will now se ee a progress bar increasing from left to right at
the bottom part of the program, see Fig.
\ref{fig:controlmeasure}.
\begin{figure}[ht]
\begin{center}
\includegraphics[width=\textwidth]{../figures/Controller_measure.png}
\end{center}
\caption{The telescope control program display when the telescope is measuring.}
\label{fig:controlmeasure}
\end{figure}

{\bf Note:} The telescope will measure for twice the specified integration
time.  This is because in addition to measure on the target (the signal you
want), the telescope also needs to measure itself (how the receiver disturbs
the signal from space). This means that if you select a \emph{Target
integration time} of 20 seconds, it will take approximately 40 seconds for the
measurement to complete.  

\section{Measurement results}
\label{sect:inspect}
When a measurement is completed the resulting spectra is stored temporarily
within the control program.  To look at your measured spectra, click on the tab
\emph{Measured spectra} at the top of the program window. You will see a window
looking like Fig.  \ref{fig:controlspectra}. On the left side is a list of all
spectra taken in this session (seince you started the program). On the right
side is a graph over the currently marked spectrum. This plot is useful for
a quick inspection of the data. You can zoom using the buttons below the figure,
and if you hover with the mouse pointer in the figure the values for that particular point
in the graph will appear below the plot. 
\begin{figure}[ht]
\begin{center}
\includegraphics[width=\textwidth]{../figures/Controller_spectra.png}
\end{center}
\caption{The tab \emph{Measured spectra} in the control program. To the left is
	a list of all measurements done in this session. The right plot shows the
	currently selected spectrum. You may select a different spectrum by clickin
	in the list. In the bottom left corner there is a button to save the
selected spectrum to the website data archive.}
\label{fig:controlspectra}
\end{figure}
While this is the easiest way to extract information from your measurement, it
may not be the most convenient. Instead, you may want to spend your time with
SALSA doing observations, and then do a careful analysis of your measurements
at another time. If so, you need to save your data.  In the bottom left corner
there is a button to save the selected spectrum to the online data archive on.
After saving a measurement it will be available to you at any time via the
SALSA website, see Sect.  \ref{sect:archive}. The data should appear in the archive
within seconds of pressing the upload-button, so please check that you data has indeed been
uploaded before leaving the control program. Please note that if you do not
upload your measurement it will be deleted once you exit the control program.

\section{Data analysis}
As mentioned in the previous section you may inspect your data directly within the control
program itself. This is the simples option, and in some cases it may be the right way.
However, in many cases you want to do a more careful analysis offline. In this section
we describe briefly how to download data from the archive and what software you may use to 
inspect the data offline. 

\subsection{PNG: Images}
The PNG format is an image of the spectrum just as it looked in the control
program when you pressed the upload button. This is useful for a quick look,
but is less accurate than inspecting the data in the control program since you
cannot get the exact values from the graph in a simple way. However, it may still
be accurate enough for quick estimates of, for example, the center velocities of peaks
in the spectrum.

\subsection{TXT: Textfiles}
The TXT format contains the spectrum in plain text, i.e. as list of
velocity/intensity pairs. This format can be read by many programs, for example
the free office suite LibreOffice available at http://www.libreoffice.org.
\subsubsection{Tutorial: Showing a TXT file with LibreOffice}

\subsection{FITS: A common format for astronomical data}

